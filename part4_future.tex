% Part 4: Future Development
\section{Future Development}

\begin{frame}
\frametitle{Long-Horizon Reinforcement Learning}
\begin{itemize}
    \item \textbf{Current limitation}: Most RL focuses on short-term rewards
    \item \textbf{Long-horizon RL challenges}:
    \begin{itemize}
        \item Credit assignment over many steps
        \item Sparse rewards
        \item Exploration in complex spaces
    \end{itemize}
    \item \textbf{Future directions}:
    \begin{itemize}
        \item Hierarchical RL: High-level and low-level policies
        \item Intrinsic motivation and curiosity-driven exploration
        \item Hindsight experience replay
        \item Multi-task and transfer learning
    \end{itemize}
    \item Critical for complex real-world agent applications
\end{itemize}
\end{frame}

\begin{frame}
\frametitle{Broader Agent Environments}
\begin{itemize}
    \item \textbf{Expanding agent capabilities} across diverse environments
    \item \textbf{Physical environments}:
    \begin{itemize}
        \item Robotics and embodied AI
        \item Real-world manipulation and navigation
        \item Human-robot interaction
    \end{itemize}
    \item \textbf{Digital environments}:
    \begin{itemize}
        \item Software engineering and code generation
        \item Web automation and data extraction
        \item Virtual assistants and productivity tools
    \end{itemize}
    \item \textbf{Social environments}:
    \begin{itemize}
        \item Multi-agent collaboration
        \item Human-AI teaming
        \item Adaptive communication strategies
    \end{itemize}
    \item Goal: General-purpose agents that operate across all domains
\end{itemize}
\end{frame}

\begin{frame}
\frametitle{World Models}
\begin{itemize}
    \item \textbf{Concept}: Internal model of how the world works
    \item \textbf{Benefits}:
    \begin{itemize}
        \item Predict consequences of actions
        \item Plan in model space (more efficient)
        \item Understand causal relationships
        \item Counterfactual reasoning: "What if...?"
    \end{itemize}
    \item \textbf{Approaches}:
    \begin{itemize}
        \item Learned world models from interaction data
        \item Hybrid symbolic-neural models
        \item Physics-informed neural networks
    \end{itemize}
    \item Key for embodied AI and robotics
\end{itemize}
\end{frame}

\begin{frame}
\frametitle{Continual Learning and Adaptation}
\begin{itemize}
    \item \textbf{Challenge}: Current models are static after training
    \item \textbf{Continual learning}:
    \begin{itemize}
        \item Learn from ongoing interactions
        \item Adapt to distribution shift
        \item Avoid catastrophic forgetting
    \end{itemize}
    \item \textbf{Approaches}:
    \begin{itemize}
        \item Elastic Weight Consolidation (EWC)
        \item Progressive neural networks
        \item Memory replay mechanisms
        \item Meta-learning for fast adaptation
    \end{itemize}
    \item Personalization: Adapt to individual users over time
\end{itemize}
\end{frame}

\begin{frame}
\frametitle{Summary: The Road Ahead}
\begin{itemize}
    \item \textbf{Technical frontiers}:
    \begin{itemize}
        \item Long-horizon RL and world models
        \item Broader agent environments
        \item Continual learning and adaptation
    \end{itemize}
    \item \textbf{Key challenges}:
    \begin{itemize}
        \item Scaling to real-world complexity
        \item Ensuring safety and alignment
        \item Achieving true generalization
    \end{itemize}
    \item \textbf{Vision}:
    \begin{itemize}
        \item General-purpose agents across all domains
        \item Seamless human-AI collaboration
        \item Responsible and beneficial AI systems
    \end{itemize}
    \item The journey from pretrain to agents continues...
\end{itemize}
\end{frame}
