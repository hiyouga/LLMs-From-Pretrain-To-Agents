% Part 3: The Rise of Agentic Intelligence
\section{The Rise of Agentic Intelligence}

\begin{frame}
\frametitle{From Models to Agents}
\begin{itemize}
    \item \textbf{Traditional LLMs}: Passive text generators
    \item \textbf{LLM Agents}: Active problem solvers with:
    \begin{itemize}
        \item Perception (understanding environment)
        \item Planning (decomposing tasks)
        \item Action (executing operations)
        \item Memory (maintaining state)
    \end{itemize}
    \item Evolution: Prompting $\rightarrow$ RAG $\rightarrow$ Web Search $\rightarrow$ Tools \& MCP
    \item Goal: Autonomous task completion
\end{itemize}
\end{frame}

\begin{frame}
\frametitle{Stage 1: Prompting Strategies}
\begin{itemize}
    \item \textbf{Few-shot prompting}: Few-shot exemplars
    \item \textbf{Zero-shot prompting}: Task instructions
    \item \textbf{Chain-of-Thought (CoT)}:
    \begin{itemize}
        \item ``Let's think step by step''
        \item Improved complex problem solving
    \end{itemize}
    \item \textbf{Advanced prompting techniques}:
    \begin{itemize}
        \item Reflection: Self-evaluation and refinement
        \item Planning: Task decomposition and strategy
        \item Role playing: Adopting specific personas
        \item Debating: Multi-perspective reasoning
    \end{itemize}
\end{itemize}
\end{frame}

\begin{frame}
\frametitle{Our work: RAL-Writer (arXiv'25)}
Proposes a retrieval-augmented long-text writer, combines plan-and-write strategy and context retrieval for controllable long-text generation.
\begin{center}
\includegraphics[width=0.8\textwidth]{images/ral_writer_framework.png}
\end{center}
\end{frame}

\begin{frame}
\frametitle{Stage 2: Retrieval-Augmented Generation (RAG)}
\begin{itemize}
    \item \textbf{Traditional RAG workflow}:
    \begin{enumerate}
        \item Document chunking
        \item BM25 or vector-based retrieval
        \item Prompt augmentation
        \item LLM generates response
    \end{enumerate}
    \item \textbf{From traditional RAG to agentic RAG}:
    \begin{itemize}
        \item RAG as a tool that agents can invoke
        \item Active decision-making on when to retrieve
        \item Dynamic retrieval strategies
    \end{itemize}
\end{itemize}
\end{frame}

\begin{frame}
\frametitle{Our work: EasyRAG (KDD'25 Workshop, 600 GitHub Stars)}
Enhanced Q\&A accuracy through query expansion, dual-route retrieval, LLM re-ranking, and LLM answer refinement.
\begin{center}
\includegraphics[width=0.8\textwidth]{images/easyrag_framework.png}
\end{center}
\end{frame}

\begin{frame}
\frametitle{Stage 3: Web Search}
\begin{itemize}
    \item \textbf{Comparison with RAG}:
    \begin{itemize}
        \item Web search: More proactive and dynamic
        \item RAG: More passive with pre-indexed knowledge
        \item Web search $\approx$ Agentic RAG (both use tool-use capabilities)
    \end{itemize}
    \item \textbf{Key insight}: Web search is essentially tool use
    \item \textbf{Advantages}:
    \begin{itemize}
        \item Real-time information access
        \item Broader knowledge coverage
        \item Active query formulation
    \end{itemize}
\end{itemize}
\end{frame}

\begin{frame}
\frametitle{Our work: Context9 MCP Server}
Transforming local knowledge into an LLM-ready second brain through MCP integration.
\begin{center}
\includegraphics[width=0.6\textwidth]{images/context9_framework.png}
\end{center}
\end{frame}

\begin{frame}
\frametitle{Stage 4: Tools and MCP}
\begin{itemize}
    \item \textbf{Evolution of tool use}:
    \begin{itemize}
        \item Single tool call $\rightarrow$ Multi-turn tool calls
        \item No reasoning $\rightarrow$ One-shot reasoning $\rightarrow$ Interleaved reasoning
    \end{itemize}
    \item \textbf{Expanding tool scope}:
    \begin{enumerate}
        \item Pre-defined functions
        \item Python interpreter
        \item GUI interactions (embodied AI)
        \item Full system shell access (Claude Code)
    \end{enumerate}
    \item \textbf{Model Context Protocol (MCP)}:
    \begin{itemize}
        \item Standard for connecting LLMs to data sources
        \item Unified interface for tools and integrations
    \end{itemize}
    \item \textbf{Skills}: Essentially MCP for the bash tool
\end{itemize}
\end{frame}

\begin{frame}
\frametitle{Our work: UI-TARS (arXiv'25, 9.1K GitHub Stars)}
\begin{columns}
\column{0.5\textwidth}
Breaking GUI agent benchmark records through carefully designed fine-tuning (SFT, DPO, and RL) and diverse agent environments.
\column{0.5\textwidth}
\begin{center}
\includegraphics[width=0.9\textwidth]{images/uitars_framework.png}
\end{center}
\end{columns}
\end{frame}
